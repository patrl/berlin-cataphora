%!TEX TS-program = pdflatex

\documentclass{beamer}

\usetheme[titleformat=smallcaps]{metropolis}

\usefonttheme{professionalfonts}

\usepackage[T1]{fontenc}
\usepackage[utf8]{inputenc}
\usepackage{libertine}
\usepackage{textcomp} % required to get special symbols
\usepackage[varqu,varl]{inconsolata}
\usepackage{amsthm}% must be loaded before newtxmath
\usepackage[scr=rsfso]{mathalfa}
\usepackage[cmintegrals, cmbraces, libertine,vvarbb]{newtxmath}
\usepackage{bm}% load after all math to give access to bold math
\makeatletter
\AtBeginDocument{\def\libertine@figurealign{}\libertineOsF}
\makeatother

\usepackage{linguex,qtree}

\usepackage{microtype}

\ProvidesFile{ling-common.def}[2018/09/24 Common code for the lingtex classes]

% conditionals
\RequirePackage{ifthen}

% checking for xetex/luatex
\RequirePackage{ifxetex}
\RequirePackage{ifluatex}

% trees
\PassOptionsToPackage{linguistics}{forest}
\RequirePackage{forest}

% bib stuff
\PassOptionsToPackage{
  backend=biber
, bibstyle=biblatex-sp-unified
, citestyle=sp-authoryear-comp
, url=false
, doi=false
, bibencoding=utf8}{biblatex}

\RequirePackage{biblatex}

% maths stuff
\RequirePackage{braket}
\RequirePackage{stmaryrd}
\DeclareFontFamily{U}{stmry}{}
% fix for stmaryrd optical sizes
\DeclareFontShape{U}{stmry}{m}{n}
 {
  <-5.5>  stmary5
  <5.5-6.5> stmary6
  <6.5-7.5> stmary7
  <7.5-8.5> stmary8
  <8.5-9.5> stmary9
  <9.5->  stmary10
   }{}
\RequirePackage{amsmath}

% acronyms
\PassOptionsToPackage{nolist}{acronym}
\RequirePackage{acronym}

% misc
\RequirePackage{multicol}
\RequirePackage{todonotes}
\RequirePackage{csquotes}

% necessary for custom macros
\RequirePackage{xparse}

\RequirePackage{booktabs}

% Simon Charlow's tower macro (https://gist.github.com/schar/2cd7de8af510e0cbeefb26720f389d59)
% requires the booktabs package
\NewDocumentCommand\semtower{mm}{% a 2-level semantic tower
  \begin{tabular}[c]{@{\,}c@{\,}}
    \(#1\)
    \\
    \midrule
    \(#2\)
    \\
  \end{tabular}
}
\NewDocumentCommand\tower{mmm}{% a 2-level type/category tower
  \begin{tabular}[c]{@{\,}c@{\,}}
    \(\hfil #1\ \vrule width .05em\ #2 \hfil\)
    \\
    \midrule
    \(#3\)
    \\
  \end{tabular}
}

% inquisitive diagrams stuff
\tikzstyle{index on}=[inner sep=2pt, white, circle, fill=black]
\tikzstyle{index off}=[inner sep=2pt, black, circle, draw]
\tikzstyle{index gray}=[inner sep=2pt, black, circle, fill=lightgray]
\tikzstyle{opaque}=[fill=gray,fill opacity=.1]
\tikzstyle{counter}=[densely dashed]

% macro for the interpretation function. Requires stmaryrd
\NewDocumentCommand\eval{sO{}O{}m}{%
  \IfBooleanTF#1
  {\ensuremath{\left\llbracket{#4}\right\rrbracket^{#2}_{#3}}}
  {\ensuremath{\left\llbracket\text{#4}\right\rrbracket^{#2}_{#3}}}
}

\NewDocumentCommand{\citeposs}{m}{\citeauthor{#1}'s (\citeyear{#1})}
\NewDocumentCommand{\citepossalt}{m}{\citeauthor{#1}'s \citeyear{#1}}

\NewDocumentCommand{\sub}{m}{\textsubscript{#1}}
\NewDocumentCommand{\supscr}{m}{\textsuperscript{#1}}

\NewDocumentCommand\type{m}{\ensuremath{\mathsf{#1}}}

\NewDocumentCommand\ml{m}{\ensuremath{\textsf{#1}}}

\NewDocumentCommand\dmroot{m}{\ensuremath{\sqrt{\text{#1}}}}

% requires pifont
\RequirePackage{pifont}
\NewDocumentCommand\cmark{}{\ding{51}}%
\NewDocumentCommand\xmark{}{\ding{55}}%

\NewDocumentCommand\trace{O{}}{\textit{t\textsubscript{#1}}}
\NewDocumentCommand\bracketStr{O{}m}{[\textsubscript{#1}\,{#2}\,]}
\NewDocumentCommand\powerset{m}{\ensuremath{𝒫(#1)}}
\NewDocumentCommand\strawsonEntails{}{\ensuremath{⊧_S}}
\NewDocumentCommand\nstrawsonEntails{}{\ensuremath{̸⊧_S}}
\NewDocumentCommand\fade{m}{\textcolor{gray}{#1}}
\NewDocumentCommand\bind{}{\ensuremath{≫\!\!=}}

% \RequirePackage{cleveref}
% \crefdefaultlabelformat{(#2#1#3)}

%%% Local Variables:
%%% mode: latex
%%% End:


\usecolortheme[snowy]{owl}
\setbeamerfont{title}{shape=\itshape,family=\rmfamily,series=\mdseries,size=\LARGE}
\setbeamerfont{subtitle}{shape=\itshape,family=\rmfamily,series=\mdseries,size=\large}
\setbeamerfont{section title}{shape=\itshape,family=\rmfamily,series=\mdseries,size=\LARGE}
\setbeamerfont{author}{family=\sffamily,series=\mdseries,shape=\scshape}
\setbeamerfont{date}{family=\sffamily,series=\mdseries,shape=\scshape}
\setbeamerfont{institute}{family=\sffamily,series=\mdseries,shape=\scshape}
\setbeamerfont{frametitle}{series=\mdseries}

\usepackage{acronym}
\renewcommand{\acsfont}[1]{\textsc{#1}}

\addbibresource[location=remote]{berlin-cataphora.bib}

% YS: orangeuce left and right margin to fit long examples in one line
\setbeamersize{text margin left=18pt, text margin right=18pt}
% YS: \elide
\usepackage[normalem]{ulem}
\renewcommand{\ULthickness}{2.0pt}
\newcommand{\elide}[1]{\textcolor{red!60}{$\langle$\sout{\textcolor{black}{#1}}$\rangle$}}
% YS: tikz libraries (necessary for inline arrows)
\usetikzlibrary{fit, arrows, positioning}



%\title{Binding \alert{back from the future}}
%\subtitle{Deriving Beck-effects via cyclic scope and local exhaustification}

%\author{Patrick D.\,Elliott
% \and
%Yasu Sudo}
%\institute{Asymmetries in language: presuppositions and beyond, ZAS Berlin}

\begin{document}

% YS: orangeuce space around example labels
\renewcommand{\Exlabelsep}{6pt}
\renewcommand{\SubExleftmargin}{18pt}

% YS: tikz settings for inline arrows
\tikzstyle{every picture}+=[remember picture]
\tikzstyle{every node} = [inner sep=2, outer sep=0, anchor=base, minimum height=16pt]
\tikzstyle{every path} = [->, thick]
\tikzstyle{na} = [baseline=0]


%% PDE this is broken in overleaf. Don't comment out.
%\begin{frame}
%  \maketitle
%\end{frame}




\begin{frame}{L-to-R asymmetry with indefinite antecedents}

Anaphora with indefinite antecedents shows left-to-right asymmetry.

\ex.
  \a. \tikz[na]{\node[fill=green!10](man1){A man};} came in, and \tikz[na]{\node[](he1){\textcolor{green}{\bf he}};} sat down.
  \b. \#\tikz[na]{\node[](he2){\textcolor{orange}{\bf He}};} came in, and \tikz[na]{\node[fill=orange!10](man2){a man};} sat down.

Same with donkey anaphora.

\ex.
  \a. Every [\textsubscript{NP} man who had \tikz[na]{\node[fill=green!10](novel1){a novel};}] [\textsubscript{VP} read \tikz[na]{\node[](it1){\textcolor{green}{\bf it}};}]
  \b. \#Every [\textsubscript{NP} man who had \tikz[na]{\node[](it2){\textcolor{orange}{\bf it}};}] [\textsubscript{VP} read \tikz[na]{\node[fill=orange!10](novel2){a novel};}]


\begin{tikzpicture}[overlay]
  \draw[green] (man1.north)  -- ++(0,.30) -| (he1.north) ;
  \draw[orange] (man2.south) --++(0,-.30) -| (he2.south);
  \draw[green] (novel1.north) --++(0,.30) -| (it1.north);
  \draw[orange] (novel2.south) --++(0,-.30) -| (it2.south);
\end{tikzpicture}

\end{frame}

\begin{frame}{Cataphora with definite antecedents}

But definite antecedents seem to allow cataphora.

\ex.
  \a. \tikz[na]{\node[fill=green!10](man3){The man};} came in, and \tikz[na]{\node[](he3){\textcolor{green}{\bf he}};} sat down.
  \b. \tikz[na]{\node[](he4){\textcolor{green}{\bf He}};} came in, and \tikz[na]{\node[fill=green!10](man4){the man};} sat down.

\medskip

\ex.
  \a. Every [\textsubscript{NP} man who had \tikz[na]{\node[fill=green!10](novel3){the novel};}] [\textsubscript{VP} read \tikz[na]{\node[](it3){\textcolor{green}{\bf it}};}]
  \b. Every [\textsubscript{NP} man who had \tikz[na]{\node[](it4){\textcolor{green}{\bf it}};}] [\textsubscript{VP} read \tikz[na]{\node[fill=green!10](novel4){the novel};}]

You might say that \Last[b] does not involve binding, but \alert{accidental coreference}.

We argue that cataphoric binding is actually possible.

\begin{tikzpicture}[overlay]
  \draw[green] (man3.north) --++(0,.30) -| (he3.north);
  \draw[green] (man4.south) --++(0,-.30) -| (he4.south);
  \draw[green] (novel3.north) --++(0,.30) -| (it3.north);
  \draw[green] (novel4.south) --++(0,-.30) -| (it4.south);
\end{tikzpicture}

\end{frame}

\begin{frame}{Roadmap}

{\bf Observations}:
\begin{itemize}
  \item Data with \alert{ellipsis with sloppy identity} show that definite antecedents can semantically bind cataphoric pronouns.
  \item Data with ellipsis and antecedents containing bound pronouns show that this cannot be due to \alert{crossover}.
\end{itemize}

{\bf Analysis}:
\begin{itemize}
  \item The existential presupposition of the definite projects and binds the pronoun.
\end{itemize}

\end{frame}

\section{Ellipsis, Binding, Cataphora}

\begin{frame}{Strict vs.\ sloppy identity}

Elided pronouns give rise to two readings {\small (Sag 1976, Williams 1977)}.

\ex. \tikz[na]{\node[fill=blue!10]{Ivan};} met \tikz[na]{\node[]{\textcolor{blue!80}{\bf his}};} student. \tikz[na]{\node[fill=green!10]{Jorge};} didn't $\begin{cases}\text{\elide{meet \tikz[na]{\node[]{\textcolor{blue!80}{\bf his}};} student}.}&\text{\textcolor{blue}{\sc strict}}\\\text{\elide{meet \tikz[na]{\node[]{\textcolor{green!80}{\bf his}};} student}.}&\text{\textcolor{green}{\sc sloppy}}\end{cases}$

\end{frame}


\begin{frame}{The Sag-Williams Generalization}

\begin{exampleblock}{The Sag-Williams Generalization}
Sloppy identity requires binding in the antecedent clause.
\end{exampleblock}

Evidence from \alert{rebinding}:

\ex.



\end{frame}

\begin{frame}{Sloppy donkeys}

Donkey anaphora licenses sloppy readings.
\ex.
  Every [\textsubscript{NP} man who had \tikz[na]{\node[fill=green!10](novel5){a Russian novel};}] [\textsubscript{VP} read \tikz[na]{\node[](it5){\textcolor{green}{\bf it}};}], and\\
  every [\textsubscript{NP} man who had \tikz[na]{\node[fill=green!10](novel6){a German novel};}] [\textsubscript{VP} did \elide{read \tikz[na]{\node[](it6){\textcolor{green}{\bf it}};}}], too.\\


\begin{tikzpicture}[overlay]
  \draw[green] (novel5.north) -- ++(0,.30) -| (it5.north);
  \draw[green] (novel6.south) -- ++(0,-.30) -| (it6.south);
\end{tikzpicture}

\end{frame}


\begin{frame}{Sloppy cataphora}
  \ex.
%  Every \textsc{linguistics} professor who wanted to read it$_x$ bought the$^x$ book by Chomsky, and\\
%  every \textsc{philosophy} professor who did \elide{want to read it$_y$} bought the book$^y$ by Yablo.
  Every {\sc linguist} who had \tikz[na]{\node[](it7){\textcolor{green}{\bf it}};} read \tikz[na]{\node[fill=green!10](chomsky){Chomsky's book};}, and\\
  every {\sc philosopher} who did \elide{read \tikz[na]{\node[](it8){\textcolor{green}{\bf it}};}} read \tikz[na]{\node[fill=green!10](yablo){Yablo's book};}.

Since the sloppy reading is available, the pronoun must be bound.

\begin{tikzpicture}[overlay]
  \draw[green] (chomsky.north) -- ++(0,.30) -| (it7.north);
  \draw[green] (yablo.south) -- ++(0,-.30) -| (it8.south);
\end{tikzpicture}

\end{frame}

\begin{frame}{Crossover}


\end{frame}

\begin{frame}{Cross binding}

\ex. Everyone who wanted \textsc{Kriszta} to read it$_x$ printed out their dissertation, and\\
everyone who wanted \textsc{Robyn} to \elide{read it$_y$} printed out their article.

\end{frame}


\section{Analysis}



\begin{frame}[allowframebreaks]{References}

  \printbibliography[heading=none]

\end{frame}

\end{document}

%%% Local Variables:
%%% mode: latex
%%% TeX-master: t
%%% TeX-engine: default
%%% End:
