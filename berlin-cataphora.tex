%!TEX TS-program = pdflatex

\documentclass{beamer}

\title{Binding back \alert{to the future}}
\author{Patrick~D. Elliott and Yasu Sudo}
\date{July 2, 2019}
\institute{Asymmetries in Language: Presuppositions and beyond -- Berlin}

\usetheme[titleformat=smallcaps]{metropolis}

\usefonttheme{professionalfonts}

\usepackage[T1]{fontenc}
\usepackage[utf8]{inputenc}
\usepackage{libertine}
\usepackage{textcomp} % required to get special symbols
\usepackage[varqu,varl]{inconsolata}
\usepackage{amsthm}% must be loaded before newtxmath
\usepackage[scr=rsfso]{mathalfa}
\usepackage[cmintegrals, cmbraces, libertine,vvarbb]{newtxmath}
\usepackage{bm}% load after all math to give access to bold math
\makeatletter
\AtBeginDocument{\def\libertine@figurealign{}\libertineOsF}
\makeatother

\usepackage{linguex,qtree}

\usepackage{microtype}

\ProvidesFile{ling-common.def}[2018/09/24 Common code for the lingtex classes]

% conditionals
\RequirePackage{ifthen}

% checking for xetex/luatex
\RequirePackage{ifxetex}
\RequirePackage{ifluatex}

% trees
\PassOptionsToPackage{linguistics}{forest}
\RequirePackage{forest}

% bib stuff
\PassOptionsToPackage{
  backend=biber
, bibstyle=biblatex-sp-unified
, citestyle=sp-authoryear-comp
, url=false
, doi=false
, bibencoding=utf8}{biblatex}

\RequirePackage{biblatex}

% maths stuff
\RequirePackage{braket}
\RequirePackage{stmaryrd}
\DeclareFontFamily{U}{stmry}{}
% fix for stmaryrd optical sizes
\DeclareFontShape{U}{stmry}{m}{n}
 {
  <-5.5>  stmary5
  <5.5-6.5> stmary6
  <6.5-7.5> stmary7
  <7.5-8.5> stmary8
  <8.5-9.5> stmary9
  <9.5->  stmary10
   }{}
\RequirePackage{amsmath}

% acronyms
\PassOptionsToPackage{nolist}{acronym}
\RequirePackage{acronym}

% misc
\RequirePackage{multicol}
\RequirePackage{todonotes}
\RequirePackage{csquotes}

% necessary for custom macros
\RequirePackage{xparse}

\RequirePackage{booktabs}

% Simon Charlow's tower macro (https://gist.github.com/schar/2cd7de8af510e0cbeefb26720f389d59)
% requires the booktabs package
\NewDocumentCommand\semtower{mm}{% a 2-level semantic tower
  \begin{tabular}[c]{@{\,}c@{\,}}
    \(#1\)
    \\
    \midrule
    \(#2\)
    \\
  \end{tabular}
}
\NewDocumentCommand\tower{mmm}{% a 2-level type/category tower
  \begin{tabular}[c]{@{\,}c@{\,}}
    \(\hfil #1\ \vrule width .05em\ #2 \hfil\)
    \\
    \midrule
    \(#3\)
    \\
  \end{tabular}
}

% inquisitive diagrams stuff
\tikzstyle{index on}=[inner sep=2pt, white, circle, fill=black]
\tikzstyle{index off}=[inner sep=2pt, black, circle, draw]
\tikzstyle{index gray}=[inner sep=2pt, black, circle, fill=lightgray]
\tikzstyle{opaque}=[fill=gray,fill opacity=.1]
\tikzstyle{counter}=[densely dashed]

% macro for the interpretation function. Requires stmaryrd
\NewDocumentCommand\eval{sO{}O{}m}{%
  \IfBooleanTF#1
  {\ensuremath{\left\llbracket{#4}\right\rrbracket^{#2}_{#3}}}
  {\ensuremath{\left\llbracket\text{#4}\right\rrbracket^{#2}_{#3}}}
}

\NewDocumentCommand{\citeposs}{m}{\citeauthor{#1}'s (\citeyear{#1})}
\NewDocumentCommand{\citepossalt}{m}{\citeauthor{#1}'s \citeyear{#1}}

\NewDocumentCommand{\sub}{m}{\textsubscript{#1}}
\NewDocumentCommand{\supscr}{m}{\textsuperscript{#1}}

\NewDocumentCommand\type{m}{\ensuremath{\mathsf{#1}}}

\NewDocumentCommand\ml{m}{\ensuremath{\textsf{#1}}}

\NewDocumentCommand\dmroot{m}{\ensuremath{\sqrt{\text{#1}}}}

% requires pifont
\RequirePackage{pifont}
\NewDocumentCommand\cmark{}{\ding{51}}%
\NewDocumentCommand\xmark{}{\ding{55}}%

\NewDocumentCommand\trace{O{}}{\textit{t\textsubscript{#1}}}
\NewDocumentCommand\bracketStr{O{}m}{[\textsubscript{#1}\,{#2}\,]}
\NewDocumentCommand\powerset{m}{\ensuremath{𝒫(#1)}}
\NewDocumentCommand\strawsonEntails{}{\ensuremath{⊧_S}}
\NewDocumentCommand\nstrawsonEntails{}{\ensuremath{̸⊧_S}}
\NewDocumentCommand\fade{m}{\textcolor{gray}{#1}}
\NewDocumentCommand\bind{}{\ensuremath{≫\!\!=}}

% \RequirePackage{cleveref}
% \crefdefaultlabelformat{(#2#1#3)}

%%% Local Variables:
%%% mode: latex
%%% End:


\usecolortheme[snowy]{owl}
\setbeamerfont{title}{shape=\itshape,family=\rmfamily,series=\mdseries,size=\LARGE}
\setbeamerfont{subtitle}{shape=\itshape,family=\rmfamily,series=\mdseries,size=\large}
\setbeamerfont{section title}{shape=\itshape,family=\rmfamily,series=\mdseries,size=\LARGE}
\setbeamerfont{author}{family=\sffamily,series=\mdseries,shape=\scshape}
\setbeamerfont{date}{family=\sffamily,series=\mdseries,shape=\scshape}
\setbeamerfont{institute}{family=\sffamily,series=\mdseries,shape=\scshape}
\setbeamerfont{frametitle}{series=\mdseries}

\usepackage{acronym}
\renewcommand{\acsfont}[1]{\textsc{#1}}

\addbibresource[location=remote]{berlin-cataphora.bib}

% YS: orangeuce left and right margin to fit long examples in one line
\setbeamersize{text margin left=18pt, text margin right=18pt}
% YS: \elide
\usepackage[normalem]{ulem}
\renewcommand{\ULthickness}{2.0pt}
\newcommand{\elide}[1]{\textcolor{red!60}{$\langle$\sout{\textcolor{black}{#1}}$\rangle$}}
% YS: tikz libraries (necessary for inline arrows)
\usetikzlibrary{fit, arrows, positioning}

\begin{document}

% YS: orangeuce space around example labels
\renewcommand{\Exlabelsep}{6pt}
\renewcommand{\SubExleftmargin}{18pt}

% YS: tikz settings for inline arrows
\tikzstyle{every picture}+=[remember picture]
\tikzstyle{every node} = [inner sep=2, outer sep=0, anchor=base, minimum height=16pt]
\tikzstyle{every path} = [->, thick]
\tikzstyle{na} = [baseline=0]

\begin{frame}
 \maketitle
\end{frame}

\begin{frame}{L-to-R asymmetry with indefinite antecedents}

Anaphora with indefinite antecedents shows left-to-right asymmetry.

\ex. {\bf Cross-sentential anaphora}\bigskip
  \a. \tikz[na]{\node[fill=green!10](man1){A man};} came in, and \tikz[na]{\node[](he1){\textcolor{green}{\bf he}};} sat down.
  \b. \#\tikz[na]{\node[](he2){\textcolor{orange}{\bf He}};} came in, and \tikz[na]{\node[fill=orange!10](man2){a man};} sat down.

\bigskip

\ex. {\bf Donkey anaphora}
  \a. Every [\textsubscript{NP} man who had \tikz[na]{\node[fill=green!10](novel1){a novel};}] [\textsubscript{VP} read \tikz[na]{\node[](it1){\textcolor{green}{\bf it}};}]
  \b. \#Every [\textsubscript{NP} man who had \tikz[na]{\node[](it2){\textcolor{orange}{\bf it}};}] [\textsubscript{VP} read \tikz[na]{\node[fill=orange!10](novel2){a novel};}]


\begin{tikzpicture}[overlay]
  \draw[green] (man1.north)  -- ++(0,.30) -| (he1.north);
  \draw[orange, dashed] (man2.south) --++(0,-.30) -| (he2.south);
  \draw[green] (novel1.north) --++(0,.30) -| (it1.north);
  \draw[orange, dashed] (novel2.south) --++(0,-.30) -| (it2.south);
\end{tikzpicture}

\end{frame}

\begin{frame}{Cataphora with definite antecedents}

Definite antecedents seem to allow \alert{cataphora}.

\ex.
  \a. \tikz[na]{\node[fill=green!10](man3){The man};} came in, and \tikz[na]{\node[](he3){\textcolor{green}{\bf he}};} sat down.
  \b. \tikz[na]{\node[](he4){\textcolor{green}{\bf He}};} came in, and \tikz[na]{\node[fill=green!10](man4){the man};} sat down.

\medskip

\ex.
  \a. Every [\textsubscript{NP} man who had \tikz[na]{\node[fill=green!10](novel3){the novel};}] [\textsubscript{VP} read \tikz[na]{\node[](it3){\textcolor{green}{\bf it}};}]
  \b. Every [\textsubscript{NP} man who had \tikz[na]{\node[](it4){\textcolor{green}{\bf it}};}] [\textsubscript{VP} read \tikz[na]{\node[fill=green!10](novel4){the novel};}]

One might say that \Last[b] does not involve {\bf binding}, but {\bf accidental coreference}.

We argue that cataphoric binding is actually possible.

\begin{tikzpicture}[overlay]
  \draw[green] (man3.north) --++(0,.30) -| (he3.north);
  \draw[green] (man4.south) --++(0,-.30) -| (he4.south);
  \draw[green] (novel3.north) --++(0,.30) -| (it3.north);
  \draw[green] (novel4.south) --++(0,-.30) -| (it4.south);
\end{tikzpicture}

\end{frame}

\begin{frame}{Roadmap}

{\bf Observations}:
\begin{itemize}
  \item Data with \alert{ellipsis with sloppy identity} show that definite antecedents can semantically bind cataphoric pronouns.
  \item Data with ellipsis and antecedents containing bound pronouns show that this cannot be due to \alert{crossover}.
\end{itemize}

{\bf Analysis}:
\begin{itemize}
  \item The existential presupposition of the definite projects and binds the pronoun.
\end{itemize}

\end{frame}

\section{Ellipsis, Binding, Cataphora}

\begin{frame}{Strict vs.\ sloppy identity}

Elided pronouns give rise to two readings (\citealt{sag1976,williams_discourse_1977}).

\ex. \tikz[na]{\node[fill=blue!10](ivan1){Ivan};} met \tikz[na]{\node[](his1){\textcolor{blue!80}{\bf his}};} student. \tikz[na]{\node[fill=green!10]{Jorge};} didn't $\begin{cases}\text{\elide{meet \tikz[na]{\node[](his2){\textcolor{blue!80}{\bf his}};} student}.}&\text{\textcolor{blue}{\sc strict}}\\\text{\elide{meet \tikz[na]{\node[](his3){\textcolor{green!80}{\bf his}};} student}.}&\text{\textcolor{green}{\sc sloppy}}\end{cases}$

\end{frame}


\begin{frame}{The Sag-Williams Generalization}

\textcolor{green}{\bf The Sag-Williams Generalization:}\\
Sloppy identity requires parallel binding in the antecedent clause.

Evidence:

\ex.
*\tikz[na]{\node[fill=blue!10](ivan2){Ivan};} said [that Tanya met \tikz[na]{\node[](his4){\textcolor{blue!80}{\bf his}};} student],\\
and she said [that \tikz[na]{\node[fill=green!10](jorge2){Jorge};} did \elide{met \tikz[na]{\node[](his5){\textcolor{green!80}{\bf his}};} student} too].
\hfill \alert{Rebinding}


\ex.*\tikz[na]{\node[fill=blue!10](ivan3){Ivan};} met \tikz[na]{\node[fill=blue!10](ivan4){Ivan's};} student, and\\
\tikz[na]{\node[fill=green!10](jorge3){Jorge};} did \elide{meet \tikz[na]{\node[fill=green, fill opacity=0.1, text opacity=1]{Jorge's};} student} too. \hfill \alert{Non-pronominal expression}


\end{frame}

\begin{frame}{Sloppy donkeys}

Donkey anaphora licenses sloppy readings.
\ex.
  Every [\textsubscript{NP} man who had \tikz[na]{\node[fill=green!10](novel5){a Russian novel};}] [\textsubscript{VP} read \tikz[na]{\node[](it5){\textcolor{green}{\bf it}};}], and\\
  every [\textsubscript{NP} man who had \tikz[na]{\node[fill=green!10](novel6){a German novel};}] [\textsubscript{VP} did \elide{read \tikz[na]{\node[](it6){\textcolor{green}{\bf it}};}}], too.\\


\begin{tikzpicture}[overlay]
  \draw[green] (novel5.north) -- ++(0,.30) -| (it5.north);
  \draw[green] (novel6.south) -- ++(0,-.30) -| (it6.south);
\end{tikzpicture}

See e.g., \citet{charlow_cross-categorial_2012}.

\end{frame}


\begin{frame}{Sloppy cataphora}
  \ex.\label{sloppycataphora}
%  Every \textsc{linguistics} professor who wanted to read it$_x$ bought the$^x$ book by Chomsky, and\\
%  every \textsc{philosophy} professor who did \elide{want to read it$_y$} bought the book$^y$ by Yablo.
  Every {\sc linguist} who bought \tikz[na]{\node[](it7){\textcolor{green}{\bf it}};} read \tikz[na]{\node[fill=green!10](chomsky){Chomsky's book};}, and\\
  every {\sc philosopher} who did \elide{bought \tikz[na]{\node[](it8){\textcolor{green}{\bf it}};}} read \tikz[na]{\node[fill=green!10](yablo){Yablo's book};}.

Since the sloppy reading is available, the pronoun can be bound.

\begin{tikzpicture}[overlay]
  \draw[green] (chomsky.north) -- ++(0,.30) -| (it7.north);
  \draw[green] (yablo.south) -- ++(0,-.30) -| (it8.south);
\end{tikzpicture}

\end{frame}

\begin{frame}{Crossover and Binding}

One might wonder if the definite is taking scope over the pronoun in each sentence:
\ex.
  \tikz[na]{\node[fill=green!10](chomsky2){Chomsky's book};} Every {\sc linguist} who bought \tikz[na]{\node[](it9){\textcolor{green}{\bf it}};} read $t$, and\\
  \tikz[na]{\node[fill=green!10](yablo2){Yablo's book};} every {\sc philosopher} who did \elide{bought \tikz[na]{\node[](it10){\textcolor{green}{\bf it}};}} read $t$.

\begin{tikzpicture}[overlay]
  \draw[green] (chomsky2.north) -- ++(0,.30) -| (it9.north);
  \draw[green] (yablo2.south) -- ++(0,-.30) -| (it10.south);
\end{tikzpicture}

\pause

But the subject quantifier can bind into the definite.

\ex. {\small
\tikz[na]{\node[fill=blue!10](everyoneme){Everyone who wanted \textsc{ME} to read \phantom{\bf it}};} \hspace{-1.5em} \tikz[na]{\node[](it11){\textcolor{green}{\bf it}};} printed out \tikz[na]{\node[](their1){};}\hspace{-1ex} \tikz[na]{\node[fill=green!10](dissertation){\textcolor{blue}{\bf their} dissertation};}, and\\[1.5\baselineskip]
\tikz[na]{\node[fill=blue!10](everyoneyou){everyone who wanted \textsc{YOU} to \elide{read \phantom{\bf it}}};} \hspace{-1.9em} \tikz[na]{\node[](it12){\textcolor{green}{\bf it}};} \hspace{.25em} printed out \tikz[na]{\node[](their2){};}\hspace{-1ex} \tikz[na]{\node[fill=green!10](essay){\textcolor{blue}{\bf their} essay};}.
}

\begin{tikzpicture}[overlay]
  \draw[blue] (everyoneme.north) -- ++(0,.30) -| ([xshift=1.2em]their1.north);
  \draw[green] (dissertation.south) -- ++(0,-.30) -| (it11.south);
  \draw[blue] (everyoneyou.north) -- ++(0,.30) -| ([xshift=1.2em]their2.north);
  \draw[green] (essay.south) -- ++(0,-.30) -| (it12.south);
\end{tikzpicture}

\end{frame}


\begin{frame}{Data Summary}

  \ex.[\ref{sloppycataphora}]
%  Every \textsc{linguistics} professor who wanted to read it$_x$ bought the$^x$ book by Chomsky, and\\
%  every \textsc{philosophy} professor who did \elide{want to read it$_y$} bought the book$^y$ by Yablo.
  Every {\sc linguist} who bought \tikz[na]{\node[](it72){\textcolor{green}{\bf it}};} read \tikz[na]{\node[fill=green!10](chomsky12){Chomsky's book};}, and\\
  every {\sc philosopher} who did \elide{bought \tikz[na]{\node[](it82){\textcolor{green}{\bf it}};}} read \tikz[na]{\node[fill=green!10](yablo12){Yablo's book};}.

\begin{tikzpicture}[overlay]
  \draw[green] (chomsky12.north) -- ++(0,.30) -| (it72.north);
  \draw[green] (yablo12.south) -- ++(0,-.30) -| (it82.south);
\end{tikzpicture}


Ellipsis with sloppy cataphora shows that cataphoric binding is possible with a definite antecedent.

An indefinite antecedent doesn't allow binding:
\ex.
  Every {\sc linguist} who bought \tikz[na]{\node[](it73){\textcolor{orange}{\bf it}};} read \tikz[na]{\node[fill=orange!10](russian2){a Russian book};}, and\\
  every {\sc philosopher} who did \elide{bought \tikz[na]{\node[](it83){\textcolor{orange}{\bf it}};}} read \tikz[na]{\node[fill=orange!10](german2){a German novel};}.

\begin{tikzpicture}[overlay]
  \draw[orange, dashed] (russian2.north) -- ++(0,.30) -|  (it73.north);
  \draw[orange, dashed] (german2.south) -- ++(0,-.30) -| (it83.south);
\end{tikzpicture}


\end{frame}


\section{Analysis}

\begin{frame}{Our idea}

  \metroset{block=fill}
  \begin{block}{The problem}
    How do we account for the ability of definites to bind to their left without dispensing with the core results of dynamic semantics in the domain of anaphora with indefinite antecedents?
  \end{block}

  \begin{block}{Our solution}
Unlike orthodox dynamic binding of a definite by an indefinite, \textit{cataphora} involves binding by a \textit{presupposition}.

We can't make sense of this in orthodox dynamic theories, so we provide an extension of DPL in which we can cash this out.
  \end{block}

\end{frame}

\begin{frame}[allowframebreaks]{Binding by presupposition}

We adopt the {\bf Sauerland notation} for presuppositions:
$$\displaystyle\frac{\text{Presupposition}}{\text{Assertion}}$$

Crucially, we take both the at-issue meaning and the presupposition to be dynamic statements.

We'll write dynamic statements using FoL syntax -- our formalisation is in Dynamic Predicate Logic (DPL) (\citealt{groenendijk_dynamic_1991}), which has the same syntax as FoL. See \citet{elliottSudoLenls,elliottSudoCat} for the details.

Bear in the mind that in DPL the scope of existentials extends across conjunction.

\end{frame}

\begin{frame}{Definites}

  \begin{description}[orthodox theories]

    \item[Orthodox theories]
      Definites denote restricted variables.

      \item[Our theory]
      Definites contribute a variable to the assertion, and an existential statement to the \textit{presupposition}.

  \end{description}

  \ex. The$^a_x$ new book is sold out.
  $$\displaystyle\frac{\exists ! a[\ml{newBook}\,a] \wedge x = a}{\ml{soldOut}\,x}$$

  N.b. since the presupposition is a DPL statement, the variable \(a\) in the equality statement is bound by the existential.

\end{frame}

\begin{frame}{Extension to names and pronominals}

  Similarly, we assume that proper names and pronominals can also have existential presuppositions.

  \ex. $\text{Paul$^{a}_{x}$ sat down} \rightsquigarrow \displaystyle\frac{\exists ! a[a = x] \wedge x = Paul}{\ml{satDown}\,x}$

  \ex. $\text{He$^{a}_{x}$ sat down} = \displaystyle\frac{\exists ! a[a = x]}{\ml{satDown}\,x}$

\end{frame}

\begin{frame}{Accommodation}

  We define an accommodation operator \(\mathbb{A}\) that takes a presuppositional statement (i.e., a pair consisting of a presupposition and an assertion), and returns a presuppositionless one by dynamically sequencing the presupposition and the assertion.

  \[\mathbb{A}\left(\frac{\phi}{\psi}\right) \coloneq \displaystyle\frac{\top}{\phi \wedge \psi}\]

  In the following, we simply omit the presupposition whenever it is trivial, so for the above we just write \(\phi \wedge \psi\).

\end{frame}

\begin{frame}[allowframebreaks]{Accounting for cataphora}

  We now have everything we need to account for cross-sentential cataphora.

  \ex. He\(_{a}\) sat down. The new arrival\(^{a}\) yawned.

 What happens to the presuppositions of the individual conjuncts? We assume that they \textit{project}, i.e., the presupposition of the first conjunct is sequenced with the presupposition of the second.

 \ex. \(\displaystyle\frac{\phi}{\alpha}\text{ and }\frac{\psi}{\beta} \coloneq \frac{\phi \wedge \psi}{\alpha \wedge \beta}\)

 \framebreak

 Post-accommodation, the existential presupposition introduced by \textit{the new arrival} binds the variable introduced by \textit{he} in the assertive dimension.

 \exi.
 \a. He\(_{a}\) sat down. $\rightsquigarrow \ml{satDown}\,a$
 \b. $\text{The new arrival\(^{a}_{x}\) yawned.}\rightsquigarrow\displaystyle\frac{\exists !a[\ml{newArrival}\,a] \wedge x = a}{\ml{yawned}\,x}$

\ex. $\begin{aligned}[t]
  &\text{He\(_{a}\) sat down. The new arrival\(^{a}_{x}\) yawned.}\\ &\rightsquigarrow\mathbb{A}\left(\displaystyle\frac{\exists ! a[\ml{newArrival}\,a] \wedge x = a}{\ml{satDown}\,a \wedge \ml{yawned}\,a}\right)\\
&\rightsquigarrow \exists ! a[\ml{newArrival}\,a] \wedge x =a \wedge \ml{satDown}\,a \wedge \ml{yawned}\,x\end{aligned}$

\end{frame}

\begin{frame}[allowframebreaks]{Cataphora with indefinite antecedents}

  We predict -- correctly in the majority of cases -- that cataphora with indefinite antecedents is disallowed.

  \exi.
  \a. If a farmer\(^{x}\) owns a donkey\(^{y}\) he\(_{x}\) beats it\(_{y}\).
  \b. *If he\(_{x}\) owns it\(_{y}\), a farmer\(^{x}\) beats a donkey\(^{y}\).

  \framebreak

  \citet[p.\,192]{chierchia_dynamics_1995} observes that cataphora with indefinite antecedents is surprisingly good in certain cases (see also \citealt{barkerShan2008}):

  \ex. If John overcooks it\(_{a}\), a hamburger\(^{a}\) usually tastes bad.

  \framebreak

  We think that there is something else going on here. Notice that cataphora with indefinite antecedents becomes bad in an \textit{episodic} context.

  \ex. *If John overcooks it\(_{a}\), a hamburger\(^{a}\) tastes bad.

  We suspect that it's not a coincidence that apparent cataphora with indefinite antecedents seem to be licensed wherever the indefinite antecedent can receive a \textit{generic} reading.

  We think that this case involves a reading of \textit{a hamburger} under which it is essentially a definite picking out a kind, although this is still a matter for future research.

\end{frame}

\begin{frame}[allowframebreaks]{Prediction: local satisfaction bleeds cataphora}

  We predict that in cases where the existential presupposition associated with a definite antecedent can be locally satisfied, it fails to license cataphora.

  First, observe that in a conditional statement, when the preupposition of the consequent is contextually entailed by the antecedent, the conditional statement is globally presuppositionless.

  \ex. If Chomsky is active, then the new Chomsky book is sold out.

  \framebreak

  We predict therefore that cataphora should be impossible in the following sentence:

  \ex. Every student who pre-ordered it\(_{a}\) knows that\\
  [If Chomsky is active, then his new book\(^{a}\) is sold out].

  We're not sure about the facts here, so this is a matter for future research.

\end{frame}

\section{Conclusion}

\begin{frame}[allowframebreaks]{Summing up}

  \begin{itemize}

      \item Empirically, \textit{cataphoric sloppy donkeys} provide evidence for genuinely cataphoric semantic binding.

      \item There is a natural tension with arguably the most successful theory of anaphora -- dynamic semantics -- which is tailored to block semantic binding that proceeds \textit{backwards}.

      \item Our goal was to account for cataphora without jettisoning the results of dynamic semantics in the domain of anaphora.

      \framebreak

    \item Our hunch was that apparent cataphora with definite antecedents involves anaphora to the \textit{presupposition} introduced by the definite.

    \item In order to cash out this intuition, we sketched a presuppositional variant of DPL, according to which presuppositions themselves are dynamic statements, and therefore can give rise to genuine dynamic binding.

    \item There are surely further ramifications of this move. We leave a thorough exploration of the properties of this system to future work.

  \end{itemize}

\end{frame}

\begin{frame}{Acknowledgements and thanks}

  Thanks to audiences at LENLS Yokohama, Paris, and to Simon Charlow and Ezra Keshet for valuable comments and feedback.

  If you have any follow-up questions, you can email us at:

  \begin{itemize}

      \item \texttt{\alert{y.sudo@ucl.ac.uk}}

      \item \texttt{\alert{elliott@leibniz-zas.de}}

  \end{itemize}

\end{frame}

\begin{frame}[allowframebreaks]{References}

\setbeamertemplate{bibliography item}{}
 \printbibliography[heading=none]

\end{frame}

\end{document}

%%% Local Variables:
%%% mode: latex
%%% TeX-master: t
%%% TeX-engine: default
%%% End:
