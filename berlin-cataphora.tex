%!TEX TS-program = pdflatex

\documentclass{beamer}

\usetheme[titleformat=smallcaps]{metropolis}

\usefonttheme{professionalfonts}

\usepackage[T1]{fontenc}
\usepackage[utf8]{inputenc}
\usepackage{libertine}
\usepackage{textcomp} % required to get special symbols
\usepackage[varqu,varl]{inconsolata}
\usepackage{amsthm}% must be loaded before newtxmath
\usepackage[scr=rsfso]{mathalfa}
\usepackage[cmintegrals, cmbraces, libertine,vvarbb]{newtxmath}
\usepackage{bm}% load after all math to give access to bold math
\makeatletter
\AtBeginDocument{\def\libertine@figurealign{}\libertineOsF}
\makeatother

\usepackage{linguex,qtree}

\usepackage{microtype}

\input{ling-common.def}

\usecolortheme[snowy]{owl}
\setbeamerfont{title}{shape=\itshape,family=\rmfamily,series=\mdseries,size=\LARGE}
\setbeamerfont{subtitle}{shape=\itshape,family=\rmfamily,series=\mdseries,size=\large}
\setbeamerfont{section title}{shape=\itshape,family=\rmfamily,series=\mdseries,size=\LARGE}
\setbeamerfont{author}{family=\sffamily,series=\mdseries,shape=\scshape}
\setbeamerfont{date}{family=\sffamily,series=\mdseries,shape=\scshape}
\setbeamerfont{institute}{family=\sffamily,series=\mdseries,shape=\scshape}
\setbeamerfont{frametitle}{series=\mdseries}

\usepackage{acronym}
\renewcommand{\acsfont}[1]{\textsc{#1}}

\addbibresource[location=remote]{berlin-cataphora.bib}

% YS: reduce left and right margin to fit long examples in one line
\setbeamersize{text margin left=18pt, text margin right=18pt}
% YS: \elide
\usepackage[normalem]{ulem}
\newcommand{\elide}[1]{\ensuremath{\textcolor{orange}{\braket{\text{\sout{\textcolor{gray}{#1}}}}}}}
% YS: tikz libraries (necessary for inline arrows)
\usetikzlibrary{fit, arrows, positioning}



%\title{Binding \alert{back from the future}}
%\subtitle{Deriving Beck-effects via cyclic scope and local exhaustification}

%\author{Patrick D.\,Elliott
% \and
%Yasu Sudo}
%\institute{Asymmetries in language: presuppositions and beyond, ZAS Berlin}

\begin{document}

% YS: Reduce space around example labels
\renewcommand{\Exlabelsep}{6pt}
\renewcommand{\SubExleftmargin}{18pt}

% YS: tikz settings for inline arrows
\tikzstyle{every picture}+=[remember picture]
\tikzstyle{every node} = [inner sep=2, outer sep=0, anchor=base, minimum height=16pt]
\tikzstyle{every path} = [->, thick]
\tikzstyle{na} = [baseline=0]


%% PDE this is broken in overleaf. Don't comment out.
%\begin{frame}
%  \maketitle
%\end{frame}




\begin{frame}{L-to-R asymmetry with indefinite antecedents}

Anaphora with indefinite antecedents shows left-to-right asymmetry.

\ex.
  \a. \tikz[na]{\node[fill=green!10](man1){A man};} came in, and \tikz[na]{\node[](he1){\textcolor{green}{\bf he}};} sat down.
  \b. \#\tikz[na]{\node[](he2){\textcolor{red}{\bf He}};} came in, and \tikz[na]{\node[fill=red!10](man2){a man};} sat down.

Same with donkey anaphora.

\ex.
  \a. Every [\textsubscript{NP} man who had \tikz[na]{\node[fill=green!10](novel1){a novel};}] [\textsubscript{VP} read \tikz[na]{\node[](it1){\textcolor{green}{\bf it}};}]
  \b. \#Every [\textsubscript{NP} man who had \tikz[na]{\node[](it2){\textcolor{red}{\bf it}};}] [\textsubscript{VP} read \tikz[na]{\node[fill=red!10](novel2){a novel};}]


\begin{tikzpicture}[overlay]
  \draw[green] (man1.north) --++(0,.30) -| (he1.north);
  \draw[red] (man2.south) --++(0,-.30) -| (he2.south);
  \draw[green] (novel1.north) --++(0,.30) -| (it1.north);
  \draw[red] (novel2.south) --++(0,-.30) -| (it2.south);
\end{tikzpicture}

\end{frame}

\begin{frame}{Cataphora with definite antecedents}

But definite antecedents seem to allow cataphora.

\ex.
  \a. \tikz[na]{\node[fill=green!10](man3){The man};} came in, and \tikz[na]{\node[](he3){\textcolor{green}{\bf he}};} sat down.
  \b. \tikz[na]{\node[](he4){\textcolor{green}{\bf He}};} came in, and \tikz[na]{\node[fill=green!10](man4){the man};} sat down.

\medskip

\ex.
  \a. Every [\textsubscript{NP} man who had \tikz[na]{\node[fill=green!10](novel3){the novel};}] [\textsubscript{VP} read \tikz[na]{\node[](it3){\textcolor{green}{\bf it}};}]
  \b. Every [\textsubscript{NP} man who had \tikz[na]{\node[](it4){\textcolor{green}{\bf it}};}] [\textsubscript{VP} read \tikz[na]{\node[fill=green!10](novel4){the novel};}]

You might say that \Last[b] does not involve binding, but \alert{accidental coreference}.

We argue that cataphoric binding is actually possible.

\begin{tikzpicture}[overlay]
  \draw[green] (man3.north) --++(0,.30) -| (he3.north);
  \draw[green] (man4.south) --++(0,-.30) -| (he4.south);
  \draw[green] (novel3.north) --++(0,.30) -| (it3.north);
  \draw[green] (novel4.south) --++(0,-.30) -| (it4.south);
\end{tikzpicture}

\end{frame}

\begin{frame}{Roadmap}

{\bf Observations}:
\begin{itemize}
  \item Data with \alert{ellipsis with sloppy identity} show that definite antecedents can semantically bind cataphoric pronouns.
  \item Data with ellipsis and antecedents containing bound pronouns show that this cannot be due to \alert{crossover}.
\end{itemize}

{\bf Analysis}:
\begin{itemize}
  \item The existential presupposition of the definite projects and binds the pronoun.
\end{itemize}

\end{frame}

\section{Ellipsis, Binding, Cataphora}

\begin{frame}{Strict vs.\ sloppy identity}

Elided pronouns give rise to two readings {\small (Sag 1976, Williams 1977)}.

\ex. \tikz[na]{\node[fill=blue!10]{Ivan};} met \tikz[na]{\node[]{\textcolor{blue!80}{\bf his}};} student. \tikz[na]{\node[fill=green!10]{Jorge};} didn't $\begin{cases}\text{\elide{meet \tikz[na]{\node[]{\textcolor{blue!80}{\bf his}};} student}.}&\text{\textcolor{blue}{\sc strict}}\\\text{\elide{meet \tikz[na]{\node[]{\textcolor{green!80}{\bf his}};} student}.}&\text{\textcolor{green}{\sc sloppy}}\end{cases}$

\end{frame}


\begin{frame}{The Sag-Williams Generalization}

\begin{exampleblock}{The Sag-Williams Generalization}
Sloppy identity requires binding in the antecedent clause.
\end{exampleblock}

Evidence from \alert{rebinding}:

\ex.



\end{frame}

\begin{frame}{Sloppy donkeys}

Crucially for our purposes, dynamic binding licenses sloppy readings.

\ex.
  Every farmer who owns a$^x$ donkey loves it$_x$.\\
  And every farmer who owns a$^y$ mule does \elide{love it$^y$}, too.


\end{frame}


\begin{frame}{Sloppy cataphora}
  \ex. Every \textsc{linguistics} professor who wanted to read it$_x$ bought the$^x$ book by Chomsky, and\\
  every \textsc{philosophy} professor who did \elide{want to read it$_y$} bought the book$^y$ by Yablo.

Since the sloppy reading is available, the pronoun must be bound.

\end{frame}

\begin{frame}{Crossover}


\end{frame}

\begin{frame}{Cross binding}

\ex. Every professor who wanted \textsc{Kriszta} to read it$_x$ printed out their dissertation, and
every professor who wanted \textsc{Robyn} to \elide{read it$_y$} printed out their article.

\end{frame}


\section{Analysis}



\begin{frame}[allowframebreaks]{References}

  \printbibliography[heading=none]

\end{frame}

\end{document}

%%% Local Variables:
%%% mode: latex
%%% TeX-master: t
%%% TeX-engine: default
%%% End:
