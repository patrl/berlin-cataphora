%!TEX TS-program = pdflatex

\documentclass{beamer}

\usetheme[titleformat=smallcaps]{metropolis}

\usefonttheme{professionalfonts}

\usepackage[T1]{fontenc}
\usepackage[utf8]{inputenc}
\usepackage{libertine}
\usepackage{textcomp} % required to get special symbols
\usepackage[varqu,varl]{inconsolata}
\usepackage{amsthm}% must be loaded before newtxmath
\usepackage[scr=rsfso]{mathalfa}
\usepackage[cmintegrals, cmbraces, libertine,vvarbb]{newtxmath}
\usepackage{bm}% load after all math to give access to bold math
\makeatletter
\AtBeginDocument{\def\libertine@figurealign{}\libertineOsF}
\makeatother

\usepackage{linguex,qtree}

\usepackage{microtype}

\input{ling-common.def}

\usecolortheme[snowy]{owl}
\setbeamerfont{title}{shape=\itshape,family=\rmfamily,series=\mdseries,size=\LARGE}
\setbeamerfont{subtitle}{shape=\itshape,family=\rmfamily,series=\mdseries,size=\large}
\setbeamerfont{section title}{shape=\itshape,family=\rmfamily,series=\mdseries,size=\LARGE}
\setbeamerfont{author}{family=\sffamily,series=\mdseries,shape=\scshape}
\setbeamerfont{date}{family=\sffamily,series=\mdseries,shape=\scshape}
\setbeamerfont{institute}{family=\sffamily,series=\mdseries,shape=\scshape}
\setbeamerfont{frametitle}{series=\mdseries}

\usepackage{acronym}
\renewcommand{\acsfont}[1]{\textsc{#1}}

\addbibresource[location=remote]{berlin-cataphora.bib}

% YS: orangeuce left and right margin to fit long examples in one line
\setbeamersize{text margin left=18pt, text margin right=18pt}
% YS: \elide
\usepackage[normalem]{ulem}
\renewcommand{\ULthickness}{2.0pt}
\newcommand{\elide}[1]{\textcolor{red!60}{$\langle$\sout{\textcolor{black}{#1}}$\rangle$}}
% YS: tikz libraries (necessary for inline arrows)
\usetikzlibrary{fit, arrows, positioning}

%\title{Binding \alert{back from the future}}
%\subtitle{Deriving Beck-effects via cyclic scope and local exhaustification}

%\author{Patrick D.\,Elliott
% \and
%Yasu Sudo}
%\institute{Asymmetries in language: presuppositions and beyond, ZAS Berlin}

\begin{document}

% YS: orangeuce space around example labels
\renewcommand{\Exlabelsep}{6pt}
\renewcommand{\SubExleftmargin}{18pt}

% YS: tikz settings for inline arrows
\tikzstyle{every picture}+=[remember picture]
\tikzstyle{every node} = [inner sep=2, outer sep=0, anchor=base, minimum height=16pt]
\tikzstyle{every path} = [->, thick]
\tikzstyle{na} = [baseline=0]


%% PDE this is broken in overleaf. Don't comment out.
%\begin{frame}
%  \maketitle
%\end{frame}


\begin{frame}{L-to-R asymmetry with indefinite antecedents}

Anaphora with indefinite antecedents shows left-to-right asymmetry.

\ex. {\bf Cross-sentential anaphora}\bigskip
  \a. \tikz[na]{\node[fill=green!10](man1){A man};} came in, and \tikz[na]{\node[](he1){\textcolor{green}{\bf he}};} sat down.
  \b. \#\tikz[na]{\node[](he2){\textcolor{orange}{\bf He}};} came in, and \tikz[na]{\node[fill=orange!10](man2){a man};} sat down.

\bigskip

\ex. {\bf Donkey anaphora}
  \a. Every [\textsubscript{NP} man who had \tikz[na]{\node[fill=green!10](novel1){a novel};}] [\textsubscript{VP} read \tikz[na]{\node[](it1){\textcolor{green}{\bf it}};}]
  \b. \#Every [\textsubscript{NP} man who had \tikz[na]{\node[](it2){\textcolor{orange}{\bf it}};}] [\textsubscript{VP} read \tikz[na]{\node[fill=orange!10](novel2){a novel};}]


\begin{tikzpicture}[overlay]
  \draw[green] (man1.north)  -- ++(0,.30) -| (he1.north);
  \draw[orange, dashed] (man2.south) --++(0,-.30) -| (he2.south);
  \draw[green] (novel1.north) --++(0,.30) -| (it1.north);
  \draw[orange, dashed] (novel2.south) --++(0,-.30) -| (it2.south);
\end{tikzpicture}

\end{frame}

\begin{frame}{Cataphora with definite antecedents}

Definite antecedents seem to allow \alert{cataphora}.

\ex.
  \a. \tikz[na]{\node[fill=green!10](man3){The man};} came in, and \tikz[na]{\node[](he3){\textcolor{green}{\bf he}};} sat down.
  \b. \tikz[na]{\node[](he4){\textcolor{green}{\bf He}};} came in, and \tikz[na]{\node[fill=green!10](man4){the man};} sat down.

\medskip

\ex.
  \a. Every [\textsubscript{NP} man who had \tikz[na]{\node[fill=green!10](novel3){the novel};}] [\textsubscript{VP} read \tikz[na]{\node[](it3){\textcolor{green}{\bf it}};}]
  \b. Every [\textsubscript{NP} man who had \tikz[na]{\node[](it4){\textcolor{green}{\bf it}};}] [\textsubscript{VP} read \tikz[na]{\node[fill=green!10](novel4){the novel};}]

One might say that \Last[b] does not involve {\bf binding}, but {\bf accidental coreference}.

We argue that cataphoric binding is actually possible.

\begin{tikzpicture}[overlay]
  \draw[green] (man3.north) --++(0,.30) -| (he3.north);
  \draw[green] (man4.south) --++(0,-.30) -| (he4.south);
  \draw[green] (novel3.north) --++(0,.30) -| (it3.north);
  \draw[green] (novel4.south) --++(0,-.30) -| (it4.south);
\end{tikzpicture}

\end{frame}

\begin{frame}{Roadmap}

{\bf Observations}:
\begin{itemize}
  \item Data with \alert{ellipsis with sloppy identity} show that definite antecedents can semantically bind cataphoric pronouns.
  \item Data with ellipsis and antecedents containing bound pronouns show that this cannot be due to \alert{crossover}.
\end{itemize}

{\bf Analysis}:
\begin{itemize}
  \item The existential presupposition of the definite projects and binds the pronoun.
\end{itemize}

\end{frame}

\section{Ellipsis, Binding, Cataphora}

\begin{frame}{Strict vs.\ sloppy identity}

Elided pronouns give rise to two readings {\small (Sag 1976, Williams 1977)}.

\ex. \tikz[na]{\node[fill=blue!10](ivan1){Ivan};} met \tikz[na]{\node[](his1){\textcolor{blue!80}{\bf his}};} student. \tikz[na]{\node[fill=green!10]{Jorge};} didn't $\begin{cases}\text{\elide{meet \tikz[na]{\node[](his2){\textcolor{blue!80}{\bf his}};} student}.}&\text{\textcolor{blue}{\sc strict}}\\\text{\elide{meet \tikz[na]{\node[](his3){\textcolor{green!80}{\bf his}};} student}.}&\text{\textcolor{green}{\sc sloppy}}\end{cases}$

\end{frame}


\begin{frame}{The Sag-Williams Generalization}

\textcolor{green}{\bf The Sag-Williams Generalization:}\\
Sloppy identity requires parallel binding in the antecedent clause.

Evidence:

\ex.
*\tikz[na]{\node[fill=blue!10](ivan2){Ivan};} said [that Tanya met \tikz[na]{\node[](his4){\textcolor{blue!80}{\bf his}};} student],\\
and she said [that \tikz[na]{\node[fill=green!10](jorge2){Jorge};} did \elide{met \tikz[na]{\node[](his5){\textcolor{green!80}{\bf his}};} student} too].
\hfill \alert{Rebinding}


\ex.*\tikz[na]{\node[fill=blue!10](ivan3){Ivan};} met \tikz[na]{\node[fill=blue!10](ivan4){Ivan's};} student, and\\
\tikz[na]{\node[fill=green!10](jorge3){Jorge};} did \elide{meet \tikz[na]{\node[fill=green, fill opacity=0.1, text opacity=1]{Jorge's};} student} too. \hfill \alert{Non-pronominal expression}


\end{frame}

\begin{frame}{Sloppy donkeys}

Donkey anaphora licenses sloppy readings.
\ex.
  Every [\textsubscript{NP} man who had \tikz[na]{\node[fill=green!10](novel5){a Russian novel};}] [\textsubscript{VP} read \tikz[na]{\node[](it5){\textcolor{green}{\bf it}};}], and\\
  every [\textsubscript{NP} man who had \tikz[na]{\node[fill=green!10](novel6){a German novel};}] [\textsubscript{VP} did \elide{read \tikz[na]{\node[](it6){\textcolor{green}{\bf it}};}}], too.\\


\begin{tikzpicture}[overlay]
  \draw[green] (novel5.north) -- ++(0,.30) -| (it5.north);
  \draw[green] (novel6.south) -- ++(0,-.30) -| (it6.south);
\end{tikzpicture}

\end{frame}


\begin{frame}{Sloppy cataphora}
  \ex.\label{sloppycataphora}
%  Every \textsc{linguistics} professor who wanted to read it$_x$ bought the$^x$ book by Chomsky, and\\
%  every \textsc{philosophy} professor who did \elide{want to read it$_y$} bought the book$^y$ by Yablo.
  Every {\sc linguist} who bought \tikz[na]{\node[](it7){\textcolor{green}{\bf it}};} read \tikz[na]{\node[fill=green!10](chomsky){Chomsky's book};}, and\\
  every {\sc philosopher} who did \elide{bought \tikz[na]{\node[](it8){\textcolor{green}{\bf it}};}} read \tikz[na]{\node[fill=green!10](yablo){Yablo's book};}.

Since the sloppy reading is available, the pronoun can be bound.

\begin{tikzpicture}[overlay]
  \draw[green] (chomsky.north) -- ++(0,.30) -| (it7.north);
  \draw[green] (yablo.south) -- ++(0,-.30) -| (it8.south);
\end{tikzpicture}

\end{frame}

\begin{frame}{Crossover and Binding}

One might wonder if the definite is taking scope over the pronoun in each sentence:
\ex.
  \tikz[na]{\node[fill=green!10](chomsky2){Chomsky's book};} Every {\sc linguist} who bought \tikz[na]{\node[](it9){\textcolor{green}{\bf it}};} read $t$, and\\
  \tikz[na]{\node[fill=green!10](yablo2){Yablo's book};} every {\sc philosopher} who did \elide{bought \tikz[na]{\node[](it10){\textcolor{green}{\bf it}};}} read $t$.

\begin{tikzpicture}[overlay]
  \draw[green] (chomsky2.north) -- ++(0,.30) -| (it9.north);
  \draw[green] (yablo2.south) -- ++(0,-.30) -| (it10.south);
\end{tikzpicture}

\pause

But the subject quantifier can bind into the definite.

\ex. {\small
\tikz[na]{\node[fill=blue!10](everyoneme){Everyone who wanted \textsc{ME} to read \phantom{\bf it}};} \hspace{-1.5em} \tikz[na]{\node[](it11){\textcolor{green}{\bf it}};} printed out \tikz[na]{\node[](their1){};}\hspace{-1ex} \tikz[na]{\node[fill=green!10](dissertation){\textcolor{blue}{\bf their} dissertation};}, and\\[1.5\baselineskip]
\tikz[na]{\node[fill=blue!10](everyoneyou){everyone who wanted \textsc{YOU} to \elide{read \phantom{\bf it}}};} \hspace{-1.9em} \tikz[na]{\node[](it12){\textcolor{green}{\bf it}};} \hspace{.25em} printed out \tikz[na]{\node[](their2){};}\hspace{-1ex} \tikz[na]{\node[fill=green!10](essay){\textcolor{blue}{\bf their} essay};}.
}

\begin{tikzpicture}[overlay]
  \draw[blue] (everyoneme.north) -- ++(0,.30) -| ([xshift=1.2em]their1.north);
  \draw[green] (dissertation.south) -- ++(0,-.30) -| (it11.south);
  \draw[blue] (everyoneyou.north) -- ++(0,.30) -| ([xshift=1.2em]their2.north);
  \draw[green] (essay.south) -- ++(0,-.30) -| (it12.south);
\end{tikzpicture}

\end{frame}


\begin{frame}{Data Summary}

  \ex.[\ref{sloppycataphora}]
%  Every \textsc{linguistics} professor who wanted to read it$_x$ bought the$^x$ book by Chomsky, and\\
%  every \textsc{philosophy} professor who did \elide{want to read it$_y$} bought the book$^y$ by Yablo.
  Every {\sc linguist} who bought \tikz[na]{\node[](it72){\textcolor{green}{\bf it}};} read \tikz[na]{\node[fill=green!10](chomsky12){Chomsky's book};}, and\\
  every {\sc philosopher} who did \elide{bought \tikz[na]{\node[](it82){\textcolor{green}{\bf it}};}} read \tikz[na]{\node[fill=green!10](yablo12){Yablo's book};}.

\begin{tikzpicture}[overlay]
  \draw[green] (chomsky12.north) -- ++(0,.30) -| (it72.north);
  \draw[green] (yablo12.south) -- ++(0,-.30) -| (it82.south);
\end{tikzpicture}


Ellipsis with sloppy cataphora shows that cataphoric binding is possible with a definite antecedent.

An indefinite antecedent doesn't allow binding:
\ex.
  Every {\sc linguist} who bought \tikz[na]{\node[](it73){\textcolor{orange}{\bf it}};} read \tikz[na]{\node[fill=orange!10](russian2){a Russian book};}, and\\
  every {\sc philosopher} who did \elide{bought \tikz[na]{\node[](it83){\textcolor{orange}{\bf it}};}} read \tikz[na]{\node[fill=orange!10](german2){a German novel};}.

\begin{tikzpicture}[overlay]
  \draw[orange, dashed] (russian2.north) -- ++(0,.30) -|  (it73.north);
  \draw[orange, dashed] (german2.south) -- ++(0,-.30) -| (it83.south);
\end{tikzpicture}


\end{frame}


\section{Analysis}


\begin{frame}{Binding with presupposition}

We claim that cataphora with definite antecedents involve binding with an accommodated existential presupposition.

We adopt the {\bf Sauerland notation} for presuppositions:
$$\displaystyle\frac{\text{Presupposition}}{\text{Assertion}}$$

\end{frame}

\begin{frame}{Binding with presupposition}
  \ex. The$^x_y$ new book is sold out.
  $$\displaystyle\frac{\text{There's a unique book $x$, and $x=y$}}{\text{$y$ is sold out}}$$

\end{frame}

\begin{frame}{Accommodation}

  \ex. \[\mathbb{A}\left(\frac{\phi}{\psi}\right) \coloneq \displaystyle\frac{\top}{\phi\land\psi}\]

\end{frame}


%\begin{frame}[allowframebreaks]{References}
%
%  \printbibliography[heading=none]
%
%\end{frame}

\end{document}

%%% Local Variables:
%%% mode: latex
%%% TeX-master: t
%%% TeX-engine: default
%%% End:
